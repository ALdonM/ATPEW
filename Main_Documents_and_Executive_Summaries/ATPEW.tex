\documentclass{article}
\usepackage[utf8]{inputenc}
\usepackage[T1]{fontenc}
\usepackage[french]{babel}
\usepackage{amsmath}
\usepackage{amssymb}
\usepackage{graphicx}
\usepackage{booktabs}
\usepackage{hyperref}
\usepackage{caption}
\usepackage{float}

\title{Théorie d'Aldon des Ondes Primordiales (ATPEW)}
\author{Michel Aldon}
\date{\today}

\begin{document}

\maketitle

\section*{Résumé Exécutif}
Une nouvelle physique fondamentale unifiant gravité, mécanique quantique et cosmologie via une onde primordiale d'énergie.

\section{Introduction}
La Théorie d'Aldon des Ondes Primordiales (ATPEW) propose une révolution conceptuelle pour unifier la gravité, la mécanique quantique et la cosmologie en une seule théorie cohérente. Contrairement aux modèles standards, ATPEW explique que l'espace, le temps, la matière et la gravité émergent d'une onde primordiale d'énergie, caractérisée par une amplitude \(\hat{A}\) et une vitesse de phase \(\hat{C}\).

\subsection{Objectifs}
Résoudre les problèmes non expliqués par les théories actuelles :
\begin{itemize}
    \item Origine de la masse et de la matière.
    \item Nature de l'énergie noire (\(\Lambda\)).
    \item Principe d'équivalence et gravité quantique.
    \item Asymétrie matière/antimatière.
\end{itemize}

\section{Postulats Fondamentaux}

\subsection{L'Onde Primordiale}
L'univers est rempli d'une onde d'énergie fondamentale, oscillant avec :
\begin{itemize}
    \item \(\hat{A}\) : Amplitude liée à la densité d'énergie (\(\rho \propto \hat{A}^2\)).
    \item \(\hat{C}\) : Vitesse de phase, déterminant le flux temporel local.
\end{itemize}

\subsection{Équations Clés}
\subsubsection{Amplitude locale}
\[
\hat{A}_{(r, T, P)} = \hat{A}_0 \cdot \sqrt{1 + \alpha \frac{T_0}{T} + \beta \frac{P}{P_0}} \cdot e^{-\gamma(T, P)} \cdot \sqrt{1 - \frac{2GM}{r \hat{C}_0^2}}
\]
où \(\gamma(T, P) = \gamma_T \frac{T_0}{T} + \gamma_P \frac{P_0}{P}\), \(\alpha = 10^{-2}\), \(\beta = 10^{-3}\), \(\gamma_T = 10^{-4}\), \(\gamma_P = 10^{-5}\).

\subsubsection{Vitesse de phase locale}
\[
\hat{C}_{\text{local}} = \hat{C}_0 \cdot \sqrt{\frac{h \nu}{m \hat{C}_0^2}} \cdot \sqrt{1 - \frac{2GM}{r \hat{C}_0^2}}
\]

\section{Mécanismes d'Émergence}
\subsection{Création de l'Espace et du Temps}
\begin{itemize}
    \item Espace : Émerge de la densité d'énergie (\(\rho \propto \hat{A}^2\)).
    \item Temps : Écoulement déterminé par \(\hat{C}_{\text{local}}\).
\end{itemize}

\subsection{Gravité et Matière}
\begin{itemize}
    \item La gravité émerge des gradients de \(\hat{A}\).
    \item La matière se forme par condensation de \(\hat{A}\).
\end{itemize}

\section{Avancées Clés d'ATPEW}
\begin{table}[h]
\centering
\begin{tabular}{ccc}
\toprule
\textbf{Problème Non Résolu} & \textbf{Solution ATPEW} & \textbf{Théories Standard} \\
\midrule
Origine de la masse & Condensation de \(\hat{A}\) en particules. & Champ de Higgs. \\
Énergie noire (\(\Lambda\)) & \(\Lambda \propto \hat{A}_{\text{min}}^2\). & Constante cosmologique arbitraire. \\
Gravité quantique & \(\hat{A}\) et \(\hat{C}\) unifient espace-temps et mécanique quantique. & Théorie des cordes. \\
\bottomrule
\end{tabular}
\caption{Comparaison des solutions ATPEW et des théories standard.}
\end{table}

\section{Prédictions et Tests Expérimentaux}
\subsection{Prédictions Uniques}
\begin{itemize}
    \item Variations de \(\hat{C}_{\text{local}}\) : Testables avec des horloges atomiques (mission ACES).
    \item Énergie noire dynamique : Compatible avec les données de Planck.
    \item Asymétrie matière/antimatière : Déséquilibre de \(\mathbf{10^{-12}}\).
\end{itemize}

\section{Applications Cosmologiques}
\subsection{Du Big Bang à Aujourd'hui}
\begin{itemize}
    \item Ère de Planck : \(\hat{A}\) maximal.
    \item Inflation : \(\hat{A}(t) \approx \hat{A}_0 e^{-Ht}\).
    \item Ère actuelle : \(\hat{A} \rightarrow \hat{A}_{\text{noir}}\).
\end{itemize}

\section{Conclusion}
ATPEW propose une vision unifiée et testable de la physique fondamentale.

\section{Figures}
\begin{figure}[H]
    \centering
    \includegraphics[width=0.8\textwidth]{cycle_cosmologique.png}
    \caption{Schéma du cycle cosmologique ATPEW : propagation, amortissement, contraction et rebond.}
    \label{fig:cycle}
\end{figure}

\begin{figure}[H]
    \centering
    \includegraphics[width=0.8\textwidth]{amplitude_phase.png}
    \caption{Schéma de l'amplitude \(\hat{A}\) et de la vitesse de phase \(\hat{C}\) en fonction de l'espace et du temps.}
    \label{fig:amplitude_phase}
\end{figure}

\end{document}
